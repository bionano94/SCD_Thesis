
\documentclass{standalone}




\begin{document}




\lstset{style=python}
	\begin{lstlisting}[language = python, caption = Segmentation pipeline, label =segmentation]
import numpy as np
from sklearn.mixture import GaussianMixture
 
def brain_segmentation ( brain, brain_mask, wm_mask, gm_mask, csf_mask, auto_mean = False, undefined = False, proba = False ):
    
    #casting of the mask for the normalization.
    OutputType = itk.Image[itk.UC, 3]
    cast_filter = itk.CastImageFilter[type(brain_mask), OutputType].New()
    cast_filter.SetInput(brain_mask)
    cast_filter.Update()
    brain_mask = cast_filter.GetOutput()
    
    #brain normalization
    norm_brain_filter = itk_gaussian_normalization (brain, brain_mask)
    
    #I change the physical space because the normalization have changed it.
    matching_filter = match_physical_spaces(norm_brain_filter.GetOutput(), brain)
    matching_filter.Update()
    brain = matching_filter.GetOutput()
    
    #mask back to float
    cast_filter = itk.CastImageFilter[type(brain_mask), OutputType].New()
    cast_filter.SetInput(brain_mask)
    cast_filter.Update()
    brain_mask = cast_filter.GetOutput()
    
    
    #linearize and indexing the brain obtaining the image array and the index array
    #(the index array is useful to build the itk label image)
    brain_array, index_array = indexing (brain, brain_mask)
    
    
    #Masking also the Probability maps
    wm_mask = negative_3d_masking(wm_mask, brain_mask)
    gm_mask = negative_3d_masking(gm_mask, brain_mask)
    csf_mask = negative_3d_masking(csf_mask, brain_mask)

    
    #INITIALIZING THE MODELS PARAMETERS    
    if  auto_mean : #if automean the proposed function will be used to find the means values
    
        if undefined :
              means = np.reshape(find_4_means(brain, csf_mask, gm_mask, wm_mask), (-1,1))
            
        else: means = np.reshape(find_means(brain, brain_mask, csf_mask, gm_mask, wm_mask), (-1,1))
                
        #Adding a check for the mean values. 
        #We should obtain mean values sufficiently different and in order csf < gm < wm.
        #if not then the mean values will be initialized by the k-means++ algorithm of Scikit-Learn
        if ( (means[0]+0.1) >= means[1] ) or ( (means[1] + 0.1) >= means[2] ) :
            means = None
    
    else: means = None #The k-means++ algorithm of Scikit-Learn will be used to initialize the means
    
    if undefined:
        n_classes = 4
        weights = find_prob_4_weights(csf_mask, gm_mask, wm_mask)
    else:
        n_classes = 3
        weights = find_prob_weights(csf_mask, gm_mask, wm_mask)
        
    #Definition of the models    
    model = GaussianMixture(
                n_components = n_classes,
                covariance_type = 'full',
                tol = 0.01,
                max_iter = 10000,
                init_params = 'k-means++',
                means_init = means,
                weights_init = weights
                )
    
    #updating the funtion to find the labels
    model.fit( np.reshape( brain_array, (-1,1) ) )
    
    #Return 3 probability masks
    if proba:
        label_array = model.predict_proba( np.reshape( brain_array, (-1,1) ) )
        wm = label_de_indexing (label_array[:,2], index_array, brain)
        gm  = label_de_indexing (label_array[:,1], index_array, brain)
        csf  = label_de_indexing (label_array[:,0], index_array, brain)
        
        return wm, gm, csf
    
    else:
        label_array = model.predict( np.reshape( brain_array, (-1,1) ))
        label_image = label_de_indexing (label_array, index_array, brain, 1.)
        return label_image

\end{lstlisting}


\end{document}