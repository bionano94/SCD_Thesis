\documentclass{standalone}

\begin{document}
\chapter*{Conclusions}\addcontentsline{toc}{chapter}{Conclusions}
\markboth{Conclusions}{Conclusions}

In this work of thesis I have developed, implemented and tested an automatic pipeline for the brain extraction and tissue segmentation for MRI head scans, and a pipeline for the refinement of the results of an automated process of labelling Silent Cerebral Infarcts in patients with Sickle Cell Disease.
The data set used in this work of thesis consists of a set of $57$ FLAIR and T1W MRI head scans provided by different medical centers in Italy. The images of the data set were already manually labelled by expert clinicians at the University of Padua.

As a first step I have performed a brain extraction. This step involves the usage of a brain mask of an already segmented head atlas which permitted first rough extraction for the brain, followed by a sequence of application of different morphological operations in combination with thresholing operations.
This step is then followed by a brain tissues segmentation implemented by the usage of an expectation maximization algorithm applied to a Gaussian mixture model whose parameters were estimated using the maps of the partial volume effect provided by the ICBM MNI 152 atlas.
The first step have permitted to obtain standardized images that the already implemented SCIs' segmentation pipeline, a U-Net ensemble, could more easily elaborate.
The second step has led to a soft segmentation of the brain, resulting in three partial volume maps for each image and this has been useful for the features extraction of the post-processing stage.

The last step has been the classification of the labels obtained by the U-Net ensemble in order to refine the results by removing the false positive outcomes of the U-Net ensemble itself.
To perform this step, different features have been extracted from the labels considering both the proper functionality of the labelling ensemble and the known anatomical characteristics of the SCIs.
A set of three classifiers has then been trained on a subset of the whole provided data set using the features extracted measuring the overlapping of the labels with the probability maps in output from the segmentation step, the physical size of the labels and the overlapping with masks manually provided by expert clinicians that delimits brain regions in which usually are localized the SCIs.

The pre-processing pipeline has been tested comparing the results obtained for both the brain extraction and the tissue segmentation to the segmentation obtained using the functions of the software FSL. 
For the whole data set has been measured the Dice similarity coefficient (DSC) of the overlapping between the obtained masks.

The mean dice similarity coefficient of the proposed brain extraction technique is $0.87 \pm 0.12$ and on some images performed better than the software using for comparison.
The tissue segmentation pipeline has shown some issues on the images with poor contrast but have obtained however a mean DSC value of $0.78 \pm 0.19$ for the white matter segmentation, $0.67 \pm 0.30$ for the grey matter and $0.66 \pm 0.24$ for the cerebrospinal fluid.
Even in this case on some volumes the proposed technique reach better results than the software used for the comparison.

The post-processing pipeline has been tested on a subset of the whole data set that has not been used during the training stage. The test set has been chosen to well represents the heterogeneity of the full data set.
The test has been performed measuring different metrics to correctly evaluate the performance. All the classifiers have shown a recall next to the maximum value ($0.97$), meaning the almost absence of false negative outcomes and a good balanced accuracy, always higher than $0.67$.
The classifier which reaches the best performances has been a random forest classifier that scores an AUC precision recall near $0.74$ and balanced accuracy of $0.74$.

The pipeline has been implemented using Python and is part of an open-source project freely available on the platform GitHub.

Further developing of this project is possible, like increasing the performances of the tissue segmentation step considering different approaches in the way the initial parameters are estimated, or to develop the pipeline of the post-processing stage to permit a refinement on the boundary and the size of the labels found by the U-Net ensemble.

In the end this project provided a suitable approach with satisfactory results for the extraction and the segmentation of the brain from head T1W MRI scans and an approach capable to improve the segmentation of the SCIs obtained with a U-Net ensemble.


\end{document}