\documentclass[]{standalone}

\begin{document}
	\begin{frame}{Data Set}{Head MRI Scans}
	\vspace{-25pt}
	\begin{columns}
		\begin{column}{0.5\textwidth}
		\begin{exampleblock}{}
		\begin{itemize}
		\footnotesize
			\item 57 MRI \textbf{T1W} and \textbf{FLAIR} head scans;
			\item Acquired from 2009 to 2020;
			\item From three different medical centers in italy;
			\item 51 with SCI evidences, 6 without lesions;
			\item High \textbf{heterogeneity} in both acquisition times and spatial resolution;
			\item Mainly \textbf{underaged} patients;
			\item Manual SCIs segmentation for each scan.
		\end{itemize}
		\end{exampleblock}
		\end{column}
		\begin{column}{0.52\textwidth}
		\begin{block}{}
		\begin{table}[h!]
			\footnotesize
			\setlength{\tabcolsep}{3pt}
			\centering
			\begin{tabular}{|c|cc|cc|}
			\hline
			\textbf{Axis} & \multicolumn{2}{c|}{\textbf{Size (pixel)}} & \multicolumn{2}{c|}{\textbf{Spacing (mm)}} \\
			  & Mean & Std. Dev. & Mean & Std. Dev. \\ \hline
			x & 256  & 0         & 0.85 & 0.12      \\
			y & 256  & 0         & 0.81 & 0.14      \\
			z & 90   & 132       & 4.34 & 1.93      \\ \hline
			\end{tabular}
		\caption*{Spatial Resolution of the images}
		\end{table}
		\vspace{-10pt}
		\end{block}
		\end{column}
	\end{columns}
	\end{frame}
\end{document}
